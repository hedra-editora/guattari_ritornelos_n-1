\raggedright

{\Formular{\textbf{FÉLIX GUATTARI} \textit{psicanalista, filósofo e militante, nasceu na França em 1930 e morreu em 1992. Trabalhou na clínica de La Borde, junto a Jean Oury. Publicou {\slsc{Psicanálise e transversalidade}}, {\slsc{A revolução molecular}}, {\slsc{As três ecologias}} e {\slsc{Caosmose}}, entre outros. Com Deleuze escreveu sobretudo {\slsc{O anti-Édipo}} e {\slsc{Mil Platôs}}, renovando o panorama da filosofia francesa.}

\textit{Conhecido sobretudo pelas noções de “micropolítica”, “esquizoanálise” e “ecosofia”, esteve
sempre atento às mutações do desejo e da subjetividade contemporânea. Interessou-se
desde cedo pelos movimentos políticos no Brasil, pelo surgimento das rádios livres na Itália,
pela rede internacional de alternativas à psiquiatria – em suma, pelo que ele mesmo
chamou de “revoluções moleculares”.}
}}


