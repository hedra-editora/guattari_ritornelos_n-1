\raggedright

{\Formular{\textbf{FÉLIX GUATTARI} \textit{foi psicanalista, filósofo e militante francês. Nasceu em 1930 e morreu em 1992. Trabalhou a vida toda na clínica de La Borde. Publicou {\slsc{Psicanálise e transversalidade}}, {\slsc{A revolução molecular}}, {\slsc{As três ecologias}}, {\slsc{Caosmose}} e {\slsc{Cartografias esquizoanalíticas}}, entre outros. Com Deleuze escreveu {\slsc{O anti-Édipo}}, {\slsc{Kafka, por uma literatura menor}}, {\slsc{Mil Platôs}} e {\slsc{O que é a filosofia?}}, renovando o panorama da filosofia francesa.}

\textit{Conhecido sobretudo pelas noções de “micropolítica”, “esquizoanálise” e “ecosofia”, esteve sempre atento às mutações do desejo e da subjetividade contemporânea. Interessou-se desde cedo pelos movimentos políticos no Brasil, pelo surgimento das rádios livres na Itália, pela rede internacional de alternativas à psiquiatria – em suma, pelo que ele mesmo chamou de “revoluções moleculares”.}
}}

